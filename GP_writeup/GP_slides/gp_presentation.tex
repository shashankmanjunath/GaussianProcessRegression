\documentclass[pdf]{beamer}
\usetheme{metropolis}

\usepackage[utf8]{inputenc} % allow utf-8 input
\usepackage[T1]{fontenc}    % use 8-bit T1 fonts
\usepackage{hyperref}       % hyperlinks
\usepackage{url}            % simple URL typesetting
\usepackage{booktabs}       % professional-quality tables
\usepackage{amsfonts}       % blackboard math symbols
\usepackage{nicefrac}       % compact symbols for 1/2, etc.
\usepackage{microtype}      % microtypography
\usepackage{amsmath}        % math notation etc
\usepackage{graphicx}       % inserting images
\usepackage{float}          % image placement
\usepackage{array}          % table alignment
\usepackage{xcolor}         % to do macro
\usepackage{adjustbox}      % Shrink stuff

\newcommand{\R}{\mathbb{R}}
\newcommand{\E}{\mathbb{E}}
\newcommand{\bx}{\boldsymbol{x}}
\newcommand{\todo}[1]{\textcolor{red}{#1}}

\graphicspath{ {../figures/} }

\title{On Gaussian Processes for Regression}
\author{
  \textbf{Jeffrey Alido} \\
  Department of Electrical and Computer Engineering \\
  Boston University \\
  \texttt{jalido@bu.edu} \\
  \and \\
  \textbf{Shashank Manjunath} \\
  Department of Electrical and Computer Engineering \\
  Boston University \\
  \texttt{manjuns@bu.edu}
}
\date{April 27, 2021}

\begin{document}

\begin{frame}
  \titlepage
\end{frame}

\begin{frame}
  \frametitle{Introduction}

  \begin{itemize}
    \item Gaussian processes are a class of machine learning models that allow us to easily incorporate prior
      observations into our data. 
    \item Example: predicting temperatures throughout a room.
    \begin{itemize}
      \item Suppose you are trying to determine the temperature at a certain point in a room, $\bx_{n+1}$
      \item You know the temperatures at points $\{\bx_1, \hdots, \bx_n\}$
      \item If $\{\bx_1, \hdots, \bx_n, \bx_{n+1}\}$ are close, the temperatures at these points will be highly
        correlated
      \item If they are far apart, the temperatures will be less correlated.
      \item We can model the $n$ known points as a multivariate Gaussian distribution with the covariance of points
        $\bx_i$ and $\bx_j$ dependent on the physical distance between the two points, then use our distribution to
        predict the temperature at $\bx_{n+1}$
    \end{itemize}
  \end{itemize}
\end{frame}

\begin{frame}
  \frametitle{Introduction}
  \begin{itemize}
    \item In a GP, we assume any new points we observe follow the same multivariate normal observed in our training data
    \item Some history of GPs 
    	\cite{rasmussen_gaussian_2006}
      \todo{add some citations}
    \begin{itemize}
      \item Blight and Ott first introducted GPs as priors over functions in 1975
      \item Gaussian Process models were first recognized as the limit of a Bayesian neural network by Mackay (1992) and
        Neal (1996)
    \end{itemize}
    \item GPs are non-parametric models (unlike models such as Neural Networks)
    \item GPs allows us to quantify uncertainty in our predictions
    \item GPs are not advantageous in that they scale poorly to large datasets
  \end{itemize}
\end{frame}

\begin{frame}
  \frametitle{Multivariate Gaussian Distributions}

  \begin{itemize}
    \item A set of univariate Gaussian random variables may be characterized jointly as a multivariate Gaussian
      distribution, with joint probability distribution fully characterized by a mean vector and a covariance matrix:

      \[
        X = \begin{bmatrix}
                 X_{1} \\
                 X_{2} \\
                 \vdots \\
                 X_{n}
               \end{bmatrix}   \sim \mathcal{N}(\boldsymbol{\mu},\Sigma)
      \]
      \begin{itemize}
        \item $\boldsymbol{\mu}$ indicates the mean vector
        \item $\Sigma$ indicates the covariance matrix whose entries describe the covariance between each pair of random
          variables
      \end{itemize}
  \end{itemize}
\end{frame}

\begin{frame}
  \frametitle{Gaussian Processes}
  \begin{itemize}
    \item A Gaussian process $f(\boldsymbol{x})$ is defined as a a random process where each set of random variable in
      the random process is has a multivariate Gaussian distribution.
    \item In mathematical notation, $f(\boldsymbol{x})$ is fully characterized by a mean function  $m(\boldsymbol{x})$
      and covariance function, $K(\boldsymbol{x},\boldsymbol{x'})$:
      \begin{gather*}
        f(\boldsymbol{x})\sim\mathcal{GP}(m(\boldsymbol{x}),K(\boldsymbol{x},\boldsymbol{x'})) \\
        K(\bx, \bx') = \E[(f(\bx) - m(\bx))(f(\bx') - m(\bx'))]
      \end{gather*}

    \item The mean function $m(\bx)$ is typically defined as zero
    \item The covariance is chosen based on some prior belief about the dataset
    \item The covariance function is analagous to a kernel function $\kappa(\cdot, \cdot)$, where each entry of the
      covariance matrix is the kernel function calculated between the corresponding points
  \end{itemize}
\end{frame}

\begin{frame}
  \frametitle{Gaussian Processes for Regression}

  \begin{itemize}
    \item While we have defined a Gaussian process, we now describe how to fit a Gaussian process (predictive
      distribution) given a training set (prior distribution) and test points
    \item Suppose we observe training data $\bx$, test data $\bx'$, and choose kernel $\kappa$. Then the mean and
      covariance functions are given by 

      \begin{gather*}
        m(\boldsymbol{x})=\kappa(\bx, \bx')^\top \left(\kappa(\bx, \bx) + \sigma_{n}^{2} I\right)^{-1}\bx \\
        K(\boldsymbol{x},\boldsymbol{x'}) = \kappa(\bx', \bx') - \kappa(\bx, \bx')^{\top}\left(\kappa(\bx, \bx) +
        \sigma_{n}^{2}I\right)^{-1}\kappa(\bx, \bx')
      \end{gather*}
  \end{itemize}

  Those interested in the derivation of the results are encouraged to consult section 2 of \cite{rasmussen_gaussian_2006}.
\end{frame}

\begin{frame}
  \frametitle{A Simple Demonstration}

  \todo{fit some GPs on sine waves or something}
\end{frame}

\begin{frame}
  \frametitle{The Boston Housing Dataset}
  \begin{itemize}
    \item Originally published in 1978\cite{harrison_hedonic_1978}
    \item 506 data points, 13 features, 1 label (median value of a house in a Boston suburb, in \$1000s)
    \item Well-suited to Gaussian processes due to small size
    \item Features detailed in \ref{table:bhd_feat}
  \end{itemize}
\end{frame}

\begin{frame}
  \begin{table}
    \centering
    \caption{Table of Boston Housing Dataset feature names and features}
    \resizebox{1.25\textheight}{!}{
      \begin{tabular}{ || m{3cm} | m{12cm} || }
        \hline
        \textbf{Feature Name} & \textbf{Feature Description} \\
        \hline \hline
        CRIM    & Per capita crime rate by town \\
        \hline
        ZN      & Proportion of residential land zoned for lots over 25,000 sq.ft. \\
        \hline
        INDUS   & Proportion of non-retail business acres per town. \\
        \hline
        CHAS    & Charles River dummy variable (1 if tract bounds river; 0 otherwise) \\
        \hline
        NOX     & Nitric oxides concentration (parts per 10 million) \\
        \hline
        RM      & Average number of rooms per dwelling \\
        \hline
        AGE     & Proportion of owner-occupied units built prior to 1940 \\
        \hline
        DIS     & Weighted distances to five Boston employment centres \\
        \hline
        RAD     & Index of accessibility to radial highways \\
        \hline
        TAX     & Full-value property-tax rate per \$10,000 \\
        \hline
        PTRATIO & Pupil-teacher ratio by town \\
        \hline
        B       & $1000(Bk - 0.63)^2$ where Bk is the proportion of Black people by town \\
        \hline
        LSTAT   & \% lower status of the population \\
        \hline
        MEDV    & Median value of owner-occupied homes in \$1000's \\
        \hline
      \end{tabular}
    }
    \label{table:bhd_feat}
  \end{table}
\end{frame}

\begin{frame}
  \frametitle{Normalization}

  \begin{itemize}
    \item We normalize our features, which leads to improved algorithm performance, using the following
      formula:

      \[
        X_{\text{feat}} = \frac{X_{\text{feat}} - \mu(X_{\text{feat}})}{\sigma(X_{\text{feat}})}
      \]
    \item We also normalize our label, then convert back to given units (value in \$1000s) after fitting the GP.
  \end{itemize}
\end{frame}

\begin{frame}
  \frametitle{Results}
\end{frame}

\begin{frame}
  \frametitle{References}
  \bibliographystyle{amsalpha}
  \bibliography{../gpr}
\end{frame}
\end{document}
