\documentclass[pdf]{beamer}
\usetheme{metropolis}

\usepackage[utf8]{inputenc} % allow utf-8 input
\usepackage[T1]{fontenc}    % use 8-bit T1 fonts
\usepackage{hyperref}       % hyperlinks
\usepackage{url}            % simple URL typesetting
\usepackage{booktabs}       % professional-quality tables
\usepackage{amsfonts}       % blackboard math symbols
\usepackage{nicefrac}       % compact symbols for 1/2, etc.
\usepackage{microtype}      % microtypography
\usepackage{amsmath}        % math notation etc
\usepackage{graphicx}       % inserting images
\usepackage{float}          % image placement
\usepackage{array}          % table alignment
\usepackage{xcolor}         % to do macro
\usepackage{adjustbox}      % Shrink stuff

\newcommand{\R}{\mathbb{R}}
\newcommand{\bx}{\boldsymbol{x}}
\newcommand{\todo}[1]{\textcolor{red}{#1}}

\graphicspath{ {../figures/} }

\title{On Gaussian Processes for Regression}
\author{
  \textbf{Jeffrey Alido} \\
  Department of Electrical and Computer Engineering \\
  Boston University \\
  \texttt{jalido@bu.edu} \\
  \and \\
  \textbf{Shashank Manjunath} \\
  Department of Electrical and Computer Engineering \\
  Boston University \\
  \texttt{manjuns@bu.edu}
}
\date{April 27, 2021}

\begin{document}

\begin{frame}
  \titlepage
\end{frame}

\begin{frame}
  \frametitle{Abstract}
  \begin{itemize}
    \item Gaussian processes emerged in machine learning as a powerful tool for regression and classification that
      provides interpretability through kernel choice and uncertainty quantification.
    \item By leveraging properties of multivariate normal distributions and Bayes’s rule, we may infer a probability
      distribution over possible functions when fitting a dataset
    \item This Bayesian framework allows flexibility through choosing a covariance function as a prior belief about the
      dataset, which can provide further insight into the trends of the training data. 
  \end{itemize}
\end{frame}

\begin{frame}
  \frametitle{Abstract}
  Our specific contributions are:
  \begin{itemize}
    \item We implement a multi-dimensional Gaussian process regressor in MATLAB, as well as several kernels for use with
      this regressor
    \item We evaluate the GP regression performance on the Boston Housing dataset, achieving root mean square errors
      comparable to those in the top 10 of released results on the Kaggle website
    \item We implement hyperparameter optimization through maximum likelihood estimation, to remove the need for
      manual tuning of the hyperparameters
  \end{itemize}
\end{frame}

\begin{frame}
  \frametitle{Gaussian Random Variables}

A random variable is a function that maps from an event space to a measurable space. The event space represents a set of
all possible outcomes that the random variable may take, and the measurable space is a probability measure between 0 and
1 (inclusive). We say that a random variable $X$ is normally distributed if the event space has a Gaussian probability
distribution, fully characterized by two parameters: a mean $\mu$ and variance $\sigma^2$:

\[
  X\sim \mathcal{N}(\mu,\sigma^2)
\]

\end{frame}

\begin{frame}
  \frametitle{Gaussian Random Variables}
For a one-dimensional Gaussian random variable, we refer to its distribution as a univariate Gaussian distribution. A
set of Gaussian random variables may be characterized jointly as a multivariate Gaussian distribution, with joint
probability distribution fully characterized by a mean vector and a covariance matrix:

\[
  X = \begin{bmatrix}
           X_{1} \\
           X_{2} \\
           \vdots \\
           X_{n}
         \end{bmatrix}   \sim \mathcal{N}(\boldsymbol{\mu},\Sigma)
\]

where $\boldsymbol{\mu}$ is the mean vector, and $\Sigma$ is the covariance matrix whose entries describe the covariance
between each pair of random variables.
\end{frame}

\begin{frame}
  \frametitle{Gaussian Processes}
A random process is essentially a collection of random variables jointly characterized  as a set or vector of random
variables with a multivariate joint probability distribution. A Gaussian process $f(\boldsymbol{x})$ is defined as a a
random process where each set of random variable in the random process is has a multivariate Gaussian distribution.
$f(\boldsymbol{x})$ is fully characterized by a mean function  $m(\boldsymbol{x})$ and covariance function,
$K(\boldsymbol{x},\boldsymbol{x'})$:

\[
  f(\boldsymbol{x})\sim\mathcal{GP}(m(\boldsymbol{x}),K(\boldsymbol{x},\boldsymbol{x'}))
\]

Typically, the mean function is zero. The kernel for the covariance is chosen based on some prior belief about the
dataset; more on kernels is discussed in Slide \ref{slide:kernels}.
\end{frame}

\begin{frame}
  \frametitle{Gaussian Processes for Regression}

Suppose we observe training data $\boldsymbol{t}$ and choose kernel $\kappa$. Then the mean and covariance functions are
given by 

\begin{gather*}
  m(\boldsymbol{x})=C_{\boldsymbol{x}\boldsymbol{t}}^\top C_{\boldsymbol{t}}^{-1}\boldsymbol{t} \\
  K(\boldsymbol{x},\boldsymbol{x'})=C_{\boldsymbol{x}\boldsymbol{x'}}-C_{\boldsymbol{x}\boldsymbol{t}}^\top
  C_{\boldsymbol{t}}^{-1}C_{\boldsymbol{x}\boldsymbol{t}}
\end{gather*}

where $C_{\boldsymbol{x}\boldsymbol{t}} = \kappa(\boldsymbol{x},\boldsymbol{t})$, $C_{\boldsymbol{t}} =
\kappa(\boldsymbol{t},\boldsymbol{t})$, and $C_{\boldsymbol{x}\boldsymbol{x'}} =
\kappa(\boldsymbol{x},\boldsymbol{x'})$.  Those interested in the derivation of the results are encouraged to consult
section 6.4.2 of \cite{bishop_pattern_2006}.
\end{frame}

\begin{frame}\label{slide:kernels}
  \frametitle{Kernels}

Covariance Functions or kernels, denoted $\kappa(\bx, \bx')$, form the core of a Gaussian process. Kernels allow
projection of input data into an alternate feature space, allowing easier separability of data in this new feature
space. Gaussian processes leverage kernels to featurize input data. In particular, if we have a function
$\Phi(\bx): \R^d \rightarrow \R^n$, we can write the kernel defined by this function as:

\[
  \kappa(\bx, \bx') = \langle \Phi(\bx), \Phi(\bx') \rangle = \Phi(\bx)^\top \Phi(\bx')
\]

In order to illustrate how kernels fit into Gaussian Processes, we will 

\todo{finish derivation of kernel trick in GP}

\end{frame}

\begin{frame}
  \frametitle{Radial Basis Function Kernel}

The Radial Basis Function (RBF) Kernel, also known as the Squared Exponential Kernel, is given by:

\[
  \kappa_{RBF}(\bx, \bx') = \sigma^2 \exp\left( - \frac{\|\bx -\bx' \|_{2}^{2}}{2 \ell^2} \right)
\]

This kernel is parametrized by two parameters, the lengthscale $\ell$ and the variance $\sigma^2$. The lengthscale
determines the width of the kernel, and the variance scales the kernel\cite{duvenaud_automatic_2014}. We provide an
images of the RBF Kernel with various lengthscales and variances in Figure~\ref{fig:square_exp_kernel}

\end{frame}

\begin{frame}
  \frametitle{Rational Quadratic Kernel}

The Rational Quadratic Kernel is another standard kernel is similar to the RBF kernel. It can be constructed from
summing RBF kernels with varying lengthscales. The kernel is given by:

\[
  \kappa_{RQ}(\bx, \bx') = \sigma^2 \left( 1 + \frac{\| \bx - \bx' \|_{2}^{2}}{2 \alpha \ell^2} \right)^{-\alpha}
\]

This kernel is parametrized by three parameters, the lengthscale $\ell$, the variance $\sigma^2$, and the lengthscale
weighting parameter $\alpha$\cite{duvenaud_automatic_2014}. We provide images of the Rational Quadratic Kernel with
various $\alpha$ values in Figure~\ref{fig:rqk}
\end{frame}

\begin{frame}
  \frametitle{Periodic Kernel}
The periodic kernel allows us to model periodic functions. The kernel is given by:

\[
  \kappa_{P}(\bx, \bx') = \sigma^{2}  \exp(- \frac{2 \sin^{2}(\pi \|\bx - \bx'\|)}{p \ell^{2}})
\]

This kernel is parametrized by two parameters, $p$ which describes the period of the function, and $\ell$ which is the
lengthscale\cite{duvenaud_automatic_2014}. We provide images of the Periodic Kernel with various $p$ values in
Figure~\ref{fig:periodic_kernel}
\end{frame}
\begin{frame}
  \frametitle{Bayesian Hyperparameter Optimization}
  \todo{@manjuns need to finish this}
\end{frame}

\begin{frame}
  \frametitle{A Simple Demonstration}

\todo{fit some GPs on sine waves or something}
\end{frame}

\begin{frame}
  \frametitle{The Boston Housing Dataset}
The Boston Housing Dataset, originally published in 1978 contains 506 data points, each containing 13 features and 1
label for regression\cite{harrison_hedonic_1978}. The dataset provides the median value of houses in Boston suburbs.
This dataset is particularly suitable for Gaussian processes, as the dataset is quite small. The label is the median
value of owner-occupied homes in \$1000s, and all other features are used for model fitting. The features included in
the dataset are given in Table \ref{table:bhd_feat}.

\end{frame}
\begin{frame}
  \begin{table}
    \centering
    \caption{Table of Boston Housing Dataset feature names and features}
    \resizebox{1.25\textheight}{!}{
      \begin{tabular}{ || m{3cm} | m{12cm} || }
        \hline
        \textbf{Feature Name} & \textbf{Feature Description} \\
        \hline \hline
        CRIM    & Per capita crime rate by town \\
        \hline
        ZN      & Proportion of residential land zoned for lots over 25,000 sq.ft. \\
        \hline
        INDUS   & Proportion of non-retail business acres per town. \\
        \hline
        CHAS    & Charles River dummy variable (1 if tract bounds river; 0 otherwise) \\
        \hline
        NOX     & Nitric oxides concentration (parts per 10 million) \\
        \hline
        RM      & Average number of rooms per dwelling \\
        \hline
        AGE     & Proportion of owner-occupied units built prior to 1940 \\
        \hline
        DIS     & Weighted distances to five Boston employment centres \\
        \hline
        RAD     & Index of accessibility to radial highways \\
        \hline
        TAX     & Full-value property-tax rate per \$10,000 \\
        \hline
        PTRATIO & Pupil-teacher ratio by town \\
        \hline
        B       & $1000(Bk - 0.63)^2$ where Bk is the proportion of Black people by town \\
        \hline
        LSTAT   & \% lower status of the population \\
        \hline
        MEDV    & Median value of owner-occupied homes in \$1000's \\
        \hline
      \end{tabular}
    }
    \label{table:bhd_feat}
  \end{table}
\end{frame}

\begin{frame}
  \frametitle{Normalization}

For the application of Gaussian Processes, we use the regression task, i.e. fitting to the MEDV feature. Prior to
fitting on the data, we normalize the data per feature. Specifically, for each feature in the dataset, we perform the
following operation:

\[
  X_{\text{feat}} = \frac{X_{\text{feat}} - \mu(X_{\text{feat}})}{\sigma(X_{\text{feat}})}
\]

where $\mu(X)$ is the mean value of that feature in the training set, and $\sigma(X)$ is the standard deviation of that
feature in the training set. We additionally normalize the MEDV feature, and convert it back to non-normalized units
before calculating our RMSE.
\end{frame}

\begin{frame}
  \frametitle{Results}
\end{frame}

\begin{frame}
  \frametitle{References}
  \bibliographystyle{amsalpha}
  \bibliography{../gpr}
\end{frame}
\end{document}
